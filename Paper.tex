%Original version in April 2002 by Antje Endemann
%
%Adapted for usage in Student Conference course by jubo (050118).
%Optimized for usage with pdflatex. For usage with plain latex,
%figures must be in Postscript (.ps or .eps files) and additional
%packages must be imported.
%Last updated 100105 by jubo
%
%This template works for outline and annotated bibliography
%(deliverable 2) without any changes.
%For usage with full and final papers (deliverables 3 and 4a/b)
%you must make several changes. All necessary changes are
%described in three specific commented sections preceded by
%`%%%%% BEGIN OF ADJUSTMENTS SECTION %%%%%'.
%The changes are in order of appearance:
%   - adjust the `\documentclass' command
%   - uncomment the abstract section
%   - adjust the `\bibliographystyle' command

%%%%% BEGIN OF 1st ADJUSTMENTS SECTION %%%%%
% You must use only one of the following `\documentclass' commands
% For the outline and annotated bibliography (deliverable 2)
% you must use the following command:
%
\documentclass[runningheads,a4paper,oribibl]{llncs}
%
% For the full and final paper (deliverables 3 and 4a/b)
% you must use the command below (and put the one above in comments).
% In other words, the 'oribib' option must be used for Deliverable 2
% but not for 3 and 4a/b.
% 
%
\documentclass[runningheads,a4paper]{llncs}
%
%%%%% END OF 1st ADJUSTMENTS SECTION %%%%%

\usepackage[T1]{fontenc}  %% needed for special characters (umlaut)
\usepackage[utf8]{inputenc}
\usepackage{graphicx}     %% for graphical things such as including pictures
\usepackage{url}          %% for proper formatting of URLs

\begin{document}

\pagestyle{headings}

\mainmatter

% \title{Guidelines for Designing Touch Interfaces for Controlling Robotic Nozzles in Critical Emergency Situations.}

\title{Can a high color contrast touch interface increase user reaction time when using a smart phone web based application.}

% response
% reaction time

% The abbreviated title will be shown in the headers of even pages.
% You should use the full title unless it is too long.

% \titlerunning{Touch Interfaces for Robotic Nozzles in Emergency Situations}

\titlerunning{asd}

\author{Albin Hübsch}

\institute{
	Department of Computing Science \\
	Umeå University, Sweden \\
	\email{id11ahh@cs.umu.se}
}

\maketitle

%%%%% BEGIN OF 2nd ADJUSTMENTS SECTION %%%%%
% Do NOT include an abstract in the outline and annotated bibliography
% (deliverable 2). For the full and final paper (deliverables 3 and 4a/b),
% you must uncomment the following two commands and place your text for
% the abstract in between them.
%
\begin{abstract}
Here goes paper abstract
\end{abstract}
%
%%%%% END OF 2nd ADJUSTMENTS SECTION %%%%%

\section{Introduction}
The main goal with this paper is to evaluate if color contrast has a significant impact on reaction time and usability in web based smart-phone applications. Consumer touch screen devices such as smart-phones has rapidly increased in amount and availability recent years. The touch screen technology have made great advances ~\cite{jennings2013touch} and it is more frequently used as a way of receiving user input. When technology moves to a broader audience, higher demands on usability needs to be set~\cite{gong2004guidelines}. 

=== OLD BELOW

The main goal with this paper is to evaluate if color contrast has a significant impact on usability when using a touch screen smart phone application for controlling robotic nozzles in emergency situations. From the results we will suggest four design guidelines to be used when designing applications for similar use or in similar situations. Consumer touch screen devices such as smart phones has increased in amount rapidly recent years and the touch screen technology have made great advances~\cite{jennings2013touch}. Since these devices now are available to almost everyone, higher demands on usability needs to be set~\cite{paredes2014sosphone}.

Unifire, a company that builds robotic nozzles for water cannons has just introduced a controlling system to steer the nozzles from an application that can run on almost all touch screen smart phones (2015). But the use cases of this application, often in different emergency situations sets high demands on great usability. Defining great usability as fast and low on user input errors. A simplified version of this nozzle control application was modified to be used in these test.

\section{Method}
In order to be able to test if color contrast has an substantial impact on the usability of a touch user interface we have two versions of the same simplified application for controlling robotic nozzles. The only difference between the two versions is that the first one has the characteristics of low contrast and the second one high contrast. We conducted an A/B test to measure the usability performance of the two versions. The usability performance was measured using two parameters, time and errors.
\begin{description}
\item[Time] is measured to be able to see if one version is faster.
\item[Errors] are calculated to see if one of the versions tend to produce more user error inputs.
\end{description}


\subsection{Designing the A/B test}
The conducted A/B test consists of two versions of the same application. One high contrast and one low contrast. Both applications are simplified versions of the robotic nozzle controlling application. We simplified the application by exchanging all icons to more basic shapes like square, circle, triangle and a rhomb. ~\ref{fig:application}) or \textit{PNG}

The application consists of 4 soft buttons, each representing a function. We designed a program to give instructions on a secondary monitor. The instructions was a series of shapes equally to the shapes in the application and the test persons where told to press the representative shape in the application as fast as they could to simulate an stressful situation. If they pressed the wrong button they where told just to continue and press the right one. Both the application and the instruction program are logging timestamps for all input and counting user input errors.

Each version of the application was tested on [NUMBER PLACEHOLDER] persons all within the same age group (20-30) and with a variety of backgrounds.

\subsection{Evaluation of the A/B test}
This section will in short describe the evaluation process of the A/B test. How the test results were collected and what parameters we have had focus on during the test.

\section{Result}
All results produced from the tests are here presented with clear numbers, conclusions and figures/graphics.

\subsection{Evaluation}
The evaluation itself and a text about the evaluation results.

\section{Discussion}
In this section we will discuss the results from the tests. We will discuss what the results mean and how they should and could be interpreted.

\subsection{Conclusions and Guidelines}
Here we will present our conclusions based on the resulting outcome from the tests. We will also present our four guidelines on how to design a touch interface compatible for emergency situations. These guidelines will be created based on the test results.

Below follows our guidelines that we have designed from our test results. The guidelines can be used when designing a 

\subsection{Drawbacks and Limitations}
This paper should not be used as a scientific foundation for any kind of argument. There are to many parameters affecting the results. Time limitation during the research period had a big influence on the credibility of the whole paper. We would like to emphasize some of the biggest drawbacks in the research and the paper by listing them below. \emph{Note that it is highly possible that there exists drawbacks not mentioned here}.

\begin{description}
	\item[Our group of test subjects was to small] To be able to get a more statistical valid result the test group has to be bigger.
	\item[Possibly bad simulation of emergency situation] Due to the limited time reserved for this paper and its research a true simulation of an emergency situation could not be set. We did our best but it could had been done better.
	\item[Many affecting parameters] The prototype is simple in its appearance but there are still many parameters that can affect the user interaction. Icons, icon size, optimal number of soft-buttons etc. All this, and more, should have been researched if done again. 
\end{description}

\subsection{Future Work}
Due to extremely limited time given for this paper there are many things that could be improved, made differently or simply just continued to be worked on. We have some concrete suggestions that could be a case for future work.

\begin{itemize}
	\item Extend the test group with more subjects. The patterns that slightly appeared in our results could with a bigger test group be more significant.
	\item Make the same tests but with a better simulation of an emergency situation. Our proposal to this is to stage an emergent situation and make the test person interact with the complete situation.
	\item Our collected test data can be downloaded and used freely to investigate other aspect not mentioned in this paper. One proposal is to look at the different shapes and try to detect possible error patterns between them.
\end{itemize}

%%%%% BEGIN OF 3rd ADJUSTMENTS SECTION %%%%%
% For the outline and annotated bibliography (deliverable 2)
% you must use the following two commands:
%
\nocite{*}  % Includes ALL entries from the .bib-file, even if they are
            % not '\cite{}'-ed in the text above
% \bibliographystyle{plain-annote}
%
% For the full and final paper (deliverables 3 and 4a/b)
% use ONLY the following commands (i.e. put the commands above into
% comments and uncomment the command below):
%
\bibliographystyle{splncs}
%
%%%%% END OF ADJUSTMENTS SECTION %%%%%

\bibliography{Bibliography}

\end{document}
