%Original version in April 2002 by Antje Endemann
%
%Adapted for usage in Student Conference course by jubo (050118).
%Optimized for usage with pdflatex. For usage with plain latex,
%figures must be in Postscript (.ps or .eps files) and additional
%packages must be imported.
%Last updated 100105 by jubo
%
%This template works for outline and annotated bibliography
%(deliverable 2) without any changes.
%For usage with full and final papers (deliverables 3 and 4a/b)
%you must make several changes. All necessary changes are
%described in three specific commented sections preceded by
%`%%%%% BEGIN OF ADJUSTMENTS SECTION %%%%%'.
%The changes are in order of appearance:
%   - adjust the `\documentclass' command
%   - uncomment the abstract section
%   - adjust the `\bibliographystyle' command

%%%%% BEGIN OF 1st ADJUSTMENTS SECTION %%%%%
% You must use only one of the following `\documentclass' commands
% For the outline and annotated bibliography (deliverable 2)
% you must use the following command:
%
% \documentclass[runningheads,a4paper,oribibl]{llncs}
%
% For the full and final paper (deliverables 3 and 4a/b)
% you must use the command below (and put the one above in comments).
% In other words, the 'oribib' option must be used for Deliverable 2
% but not for 3 and 4a/b.
% 
%
\documentclass[runningheads,a4paper]{llncs}
%
%%%%% END OF 1st ADJUSTMENTS SECTION %%%%%

\usepackage[T1]{fontenc}  %% needed for special characters (umlaut)
\usepackage[utf8]{inputenc}
\usepackage{graphicx}     %% for graphical things such as including pictures
\usepackage{url}          %% for proper formatting of URLs

% Albins Packages
\usepackage{array}

\begin{document}

\pagestyle{headings}

\mainmatter

% \title{Guidelines for Designing Touch Interfaces for Controlling Robotic Nozzles in Critical Emergency Situations.}

\title{Can a High Color Contrast Touch Interface Increase User Reaction Time when Using a Smart Phone Web Based Application.}

% response
% reaction time

% The abbreviated title will be shown in the headers of even pages.
% You should use the full title unless it is too long.

% \titlerunning{Touch Interfaces for Robotic Nozzles in Emergency Situations}

\titlerunning{Can High Color Contrast Increase User Reaction Time}

\author{Albin Hübsch}

\institute{
	Department of Computing Science \\
	Umeå University, Sweden \\
	\email{id11ahh@cs.umu.se}
}

\maketitle

%%%%% BEGIN OF 2nd ADJUSTMENTS SECTION %%%%%
% Do NOT include an abstract in the outline and annotated bibliography
% (deliverable 2). For the full and final paper (deliverables 3 and 4a/b),
% you must uncomment the following two commands and place your text for
% the abstract in between them.
%
\begin{abstract}
Here goes paper abstract
\end{abstract}
%
%%%%% END OF 2nd ADJUSTMENTS SECTION %%%%%

\section{Introduction}
The main goal with this paper is to evaluate if color contrast has a significant impact on usability in web based smart-phone applications. Consumer touch screen devices such as smart-phones have rapidly increased in amount and availability recent years. The touch screen technology have made great advances~\cite{jennings2013touch} and it is frequently used as a way of receiving user input. When this technology moves to a broader audience, higher demands on usability needs to be set~\cite{gong2004guidelines}. To further explore how usability can be improved using touch screen technology we have in this paper investigated if reaction time (the time it takes for a user to react on instructions, further referred to as RT in this paper) and user input errors (the number of user input errors when interacting with a system, further referred to as UIE in this paper) can reach better results by increasing color contrast within the interface. Better results are assumed as faster RTs and less number of UIEs. 

Earlier studies completed in this field has explored how we perceive different color contrast and how it can affect our reading performance~\cite{wu2003improving}. It has also been proved that chromaticity, contrast, and cone opponency in color space can affect RTs~\cite{mckeefry2003simple}. Many best practices for Mobile development have also shown that "requirements for sufficient color contrast"~\cite{marcus2013design} must be meet to best exaggerate the content. But still the question remains, if a high color contrast interface can increase the the RT and decrease user IE.

High RTs and low user IE are especially desirable when designing user interfaces for situations with high demands on quick user input and low error tolerance such as in emergency situations. The results of this paper can be used by any designer or developer.

\section{Method}
In order to be able to test if color contrast has a substantial impact on the RT and UIE of a touch user interface we have designed a simple web application for an Android smart-phone. The application exists in two versions, one with low color contrast (referred to as LCC in this paper) and one with high color contrast (referred to as HCC in this paper) as can be seen in Fig~\ref{fig:application}. We conducted an A/B test to measure the performance between the versions.

\begin{figure}
	\centering
	\includegraphics[width=\textwidth]{application}
	\caption{Our two versions of the application designed with different color contrast.
	\label{fig:application}}
\end{figure}

We designed the application by following design guidelines given here~\cite{hoober2011designing}~\cite{johnson2013designing}~\cite{gong2004guidelines}~\cite{norman2013design}. These guidelines can be seen as a set of rules to follow when designing any kind of user interface. We used guidelines to get proper sized buttons, good spacing between content and best positioning of navigation.

\subsection{Designing the A/B test}
We conducted an A/B test to measure the differences in RT and UIE between the two versions. An A/B test is a commonly used method by developers and designers to test differences in performance between applications when the differentiated factors are known~\cite{johnson2013designing}. Where in this case our known factor is the contrast difference. 

The application consists of 4 buttons\footnote{On screen button}, each button representing a function. To differentiate each button from the others we put a unique shape in each button. The shapes we used where a, square, circle, triangle and a rhomb (Fig~\ref{fig:application}).

To be able to give the test persons consistent instructions through the complete test we designed a program that presented the instructions on a secondary screen. 
\begin{figure}
	\centering
	\includegraphics[width=\textwidth]{instructions}
	\caption{Our program designed to present user instructions.
	\label{fig:instructions}}
\end{figure}
The instructions were a series of 20 instructions with shapes equal to the shapes in the application buttons. The test persons were told to press the representative shape in the application as fast as they could in order to measure the reaction time. If they pressed the wrong button they were told to continue the test by pressing the right button, the application registered this as an UIE. Each test was performed indoors with varying surroundings such as different lightning conditions and noise levels. We tested each version of the application on 5 persons, all within the same age group (20-30) and with a variety of backgrounds. In total we got 10 test results with a gender split of 50\% women and 50\% men that we later grouped into LCC and HCC. We analyzed them against each other to measure if any differences in RT or UIE between the two versions could be statistically confirmed.

\subsection{Evaluation of the A/B test}
Our test data received from the tests consisted of UNIX timestamps\footnote{More info at: \url{http://www.unixtimestamp.com/}} in milliseconds. Each time the user interacted with the application it registered which button that was pressed and saved it together with a UNIX timestamp. Equally, every time the instruction program showed a new symbol it logged the symbol together with a UNIX timestamp. As a result we received two data sets of symbols with corresponding timestamps for each test person, resulting in a total of approximately 400 records. Although the data was hard to interpret due to its compact look, a python script was written to structure the resulting timestamps into a more readable format. We calculated each difference (RT) in time between when the instruction was given and when user interaction registered. Out of this we could calculate the mean RT for both groups LCC and HCC. The calculation of RT did not take in consideration if user input was a no match. It was still considered as an reaction.

IEs were counted every time the user pressed a non matching object compared to the one given in the instructions. All IEs in each group, LCC and HCC, were counted to see if one of the groups produced more UIEs.

\section{Result}
From our results we can not draw any statistically significant differences in reaction time between the two groups high color contrast and low color contrast. This is most probably due to corrupt test data, a result from limitations in web application technology, see section \ref{sec:discussion}. With user input errors we found that the low color contrast group produced significantly more user input errors than the hight color contrast group.

\subsection{Reaction Times (RT)}
Table~\ref{tab:groupRT} shows our average RT and standard deviation for each group. According to the results people in HCC had insignificant slower RTs compared to the ones in LCC.

\begin{table}[]
	\centering
	\setlength{\tabcolsep}{1em}
	\setlength\extrarowheight{1em}
	\begin{tabular}{l|l|l}
		\textbf{} & \textbf{Average RT} & \textbf{Standard Deviation} \\ \hline
		\textbf{HC} & 783 ms & 969 ms \\ \hline
		\textbf{LC} & 550 ms & 1393 ms
	\end{tabular}
	\caption{Average reaction time and the standard deviation for each group, low color contrast and high color contrast.}
	\label{tab:groupRT}
\end{table}

To validate that the groups RTs are insignificantly different we assumed the two data sets where normal distributed and performed a two tailed T-Test. The T-Test produced a p-value of 0.17 which is bigger than a desired error margin ($\alpha$) of 5\%. The T-Test is therefore proving that no differences in RT exists within our data. But regardless what the t-test says our standard deviations are hinting that something might be wrong with our data (see section \ref{sec:discussion}).

\subsection{Input Errors (IE)}
None of the groups where completely free from user input errors. As can be seen in table~\ref{tab:userIE} users that were given the LCC interface produced 475\% more UIE than the HCC group.

\begin{table}[]
	\centering
	\setlength{\tabcolsep}{1em}
	\setlength\extrarowheight{1em}
	\begin{tabular}{l|l|l|l}
		\textbf{} & \textbf{Errors} & \textbf{Standard Deviation} & \textbf{Errors/User} \\ \hline
		\textbf{HC} & 4 & 1.10 & 0.8 \\ \hline
		\textbf{LC} & 19 & 3.49 & 3.8
	\end{tabular}
	\caption{Number of user IEs registered in each group and a calculation of IEs per user in each group}
	\label{tab:userIE}
\end{table}

Table~\ref{tab:userIE} shows us that the users in the HCC group generally had a more stable and consistent performance compared to the LCC group. We also performed a two tailed t-test on the UIE data set. The outcome of this test was a p-value of 0.001 which is significantly smaller than a $\alpha$ of 5\% telling us that the HCC group performed better.

\section{Discussion}\label{sec:discussion}
Our test results have shown that we can not increase the user reaction time by using a high color contrast interface when using a smart phone web based application. However we could see that the high color contrast interface was more reliable in performance when it comes to the amount of user input errors, according to our results. What we have seen in our collected test data can strongly negotiate the legibility of the results. The calculated standard deviations gave us hints on that something was wrong. We dug deeper into each test persons results and found that the data must be corrupt. We found that we had randomly received negative RTs which in theory should be completely impossible. We were prepared that the timestamps could be off sync due to individual system clocks and no boot pairing. This would have resulted in a consistent off sync. Instead we got inconsistent hick ups in our timestamps. We suspect this is a result from either hardware limitations or limitations in the JavaScript\footnote{A prototype based dynamic scripting language} engine for web applications on Android.

As for this our conclusion is that HC interfaces are the preferred method to design interfaces as they deliver a more stable user interaction pattern. Apart from this, smart phone web based applications should only be used if, sometimes slow technology, user input errors and long reaction times from the user can be accepted.

\subsection{Drawbacks and Limitations}\label{subsec:drawbacks}
The main drawback of this study is the corrupt data set. Something to have in mind when we draw any conclusions out of the results. This problem could be solved by simply rerunning the whole test using a native smart phone application instead of a web based solution.

Each test person performed only one test with a supervisory selected interface, the LCC or HCC version. To eliminate possible affecting surrounding factors each person should have done the test several times and with both the LCC and HCC interfaces. This would also have made it possible to recognize if the test persons memorized the patterns in any way.

\begin{description}
	\item[Small test group] To be able to get a more statistical valid result the test group has to be bigger. In this paper we also only assumed it was normal distributed, we did not test it.
	\item[Many affecting parameters] The prototype is simple in its appearance but there are still many parameters that can affect the users interaction and their RTs. Icons, icon size, optimal number of soft-buttons, size of hand-held device etc. All this, and more, should have been researched and taken into account if done again. 
	\item[Limited test environment] Our tests were made using an Android based smart-phone\footnote{OnePlus Two \url{https://oneplus.net/2}} system and the application were made using HTML5\footnote{HyperText Markup Language, markup standard for WWW}, CSS3\footnote{Cascading Style Sheet, styling standard for WWW} and JavaScript. We can not surely imply, without testing, that the same results would appear on an iPhone or any other smart-phone or even with a native application.
\end{description}

\subsection{Future Work}
Due to the failed data set that we received in this paper there are things that could be improved or continued to be worked on. 
We do have some concrete suggestions that could be case for future work. 

\begin{itemize}
	\item Perform tests using a native application.
	\item Extend the test group with more test subjects. It is possible to think that the patterns that slightly appeared in our results will appear even more significant with a larger group of test subjects.
	\item Our collected test data can be downloaded and used freely to investigate other aspects not mentioned in this paper. One proposal is to look at the different shapes and try to detect possible error patterns between them. Is the rhomb more frequently a case for IE compared to the other shapes?
	\item Do our results apply on other systems and techniques? Our tests are limited to the Android system and a web based application solution. It remains to answer if our results applies on all other systems. This is also a way of isolating some of the affecting parameters mentioned in~\ref{subsec:drawbacks}.
\end{itemize}

\section{Acknowledgments}
The author would like to thank the peer reviewers who provided valuable comments on both the content and structure of this paper.

%%%%% BEGIN OF 3rd ADJUSTMENTS SECTION %%%%%
% For the outline and annotated bibliography (deliverable 2)
% you must use the following two commands:
%
% \nocite{*}  % Includes ALL entries from the .bib-file, even if they are
            % not '\cite{}'-ed in the text above
% \bibliographystyle{plain-annote}
%
% For the full and final paper (deliverables 3 and 4a/b)
% use ONLY the following commands (i.e. put the commands above into
% comments and uncomment the command below):
%
\bibliographystyle{splncs}
%
%%%%% END OF ADJUSTMENTS SECTION %%%%%

\bibliography{Bibliography}

\end{document}
