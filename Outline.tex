%Original version in April 2002 by Antje Endemann
%
%Adapted for usage in Student Conference course by jubo (050118).
%Optimized for usage with pdflatex. For usage with plain latex,
%figures must be in Postscript (.ps or .eps files) and additional
%packages must be imported.
%Last updated 100105 by jubo
%
%This template works for outline and annotated bibliography
%(deliverable 2) without any changes.
%For usage with full and final papers (deliverables 3 and 4a/b)
%you must make several changes. All necessary changes are
%described in three specific commented sections preceded by
%`%%%%% BEGIN OF ADJUSTMENTS SECTION %%%%%'.
%The changes are in order of appearance:
%   - adjust the `\documentclass' command
%   - uncomment the abstract section
%   - adjust the `\bibliographystyle' command

%%%%% BEGIN OF 1st ADJUSTMENTS SECTION %%%%%
% You must use only one of the following `\documentclass' commands
% For the outline and annotated bibliography (deliverable 2)
% you must use the following command:
%
\documentclass[runningheads,a4paper,oribibl]{llncs}
%
% For the full and final paper (deliverables 3 and 4a/b)
% you must use the command below (and put the one above in comments).
% In other words, the 'oribib' option must be used for Deliverable 2
% but not for 3 and 4a/b.
% 
%
%\documentclass[runningheads,a4paper]{llncs}
%
%%%%% END OF 1st ADJUSTMENTS SECTION %%%%%

\usepackage[T1]{fontenc}  %% needed for special characters (umlaut)
\usepackage[latin1]{inputenc}
\usepackage{graphicx}     %% for graphical things such as including pictures
\usepackage{url}          %% for proper formatting of URLs

\begin{document}

\pagestyle{headings}

\mainmatter

\title{Guidelines for Designing Touch Interfaces for Controlling Robotic Nozzles in Critical Emergency Situations. \thanks{wtf}}

% The abbreviated title will be shown in the headers of even pages.
% You should use the full title unless it is too long.
\titlerunning{Touch Interfaces for Robotic Nozzles}

\author{Albin Hübsch}

\institute{
	Department of Computing Science \\
	Umeå University, Sweden \\
	\email{id11ahh@cs.umu.se}
}

\maketitle

%%%%% BEGIN OF 2nd ADJUSTMENTS SECTION %%%%%
% Do NOT include an abstract in the outline and annotated bibliography
% (deliverable 2). For the full and final paper (deliverables 3 and 4a/b),
% you must uncomment the following two commands and place your text for
% the abstract in between them.
%
% \begin{abstract}
%   Here goes the actual text of your abstract.
% \end{abstract}
%
%%%%% END OF 2nd ADJUSTMENTS SECTION %%%%%

\section{Introduction}
Introduction part will contain a purpose of the paper and a motivation for why the paper was made.

\begin{description}
	\item[Purpose of the paper] Can a well designed touch interface for controlling robotic nozzles reach a level of zero faulty interactions in critical emergency situations by increasing the color contrast of the interface?
	\item[Motivation for the paper] In critical emergency situations, like in the case of fire, a graphical touch user interface should not be the weak link on getting the job done when controlling robotic nozzles. One single press on the wrong button can be the difference between life or death.
\end{description}

\section{Method}
The main goal with this paper is to evaluate if increased color contrast in a touch screen interface for controlling robotic nozzles can result in zero faulty interactions with the touch interface in emergency situations.

\subsection{Design of an A/B test}
In order to test if increased color contrast in a touch interface can completely remove all faulty interactions an A/B test was conducted.
\emph{This section will contain a short but descriptive text about how the A/B test was prepared, built and performed. It will describe who participated}.

\subsection{Evaluation of the A/B test}
This section will in short describe the evaluation process of the A/B test. How the test results were collected and what parameters we've had focus on during the test.

\section{Result}
All results produced from the tests are here presented with clear numbers and conclusions.

\subsection{Evaluation}
The evaluation itself and a text about the evaluation results.

\section{Discussion}
In this section we'll discuss the results from the tests. We'll discuss what the results mean and how they should, and could be interpreted.

\subsection{Conclusions and Guidelines}
Here we'll present our conclusions based on the resulting outcome from the tests. We'll also present our four guidelines on how to design a touch interface compatible for emergency situations. These guidelines will be created based on the test results.

\subsection{Drawbacks and Limitations}
Present drawbacks and limitations with both the used method and the produced results. Time is limited, therefore a bullet proof test can not be performed which means the results will somewhat be limited in credibility.

\subsection{Future Work}
Due to the limited time we have on this paper there will be a lot of things that can be improved. We'll in this section give our suggestions on things that could be done in future or continued work.

%%%%% BEGIN OF 3rd ADJUSTMENTS SECTION %%%%%
% For the outline and annotated bibliography (deliverable 2)
% you must use the following two commands:
%
\nocite{*}  % Includes ALL entries from the .bib-file, even if they are
            % not '\cite{}'-ed in the text above
\bibliographystyle{plain-annote}
%
% For the full and final paper (deliverables 3 and 4a/b)
% use ONLY the following commands (i.e. put the commands above into
% comments and uncomment the command below):
%
% \bibliographystyle{splncs}
%
%%%%% END OF ADJUSTMENTS SECTION %%%%%

\bibliography{demo}

\end{document}
